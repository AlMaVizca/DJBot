%%%%%%%%%%%%%%%%%%%%%%% file typeinst.tex %%%%%%%%%%%%%%%%%%%%%%%%%
%
% This is the LaTeX source for the instructions to authors using
% the LaTeX document class 'llncs.cls' for contributions to
% the Lecture Notes in Computer Sciences series.
% http://www.springer.com/lncs       Springer Heidelberg 2006/05/04
%
% It may be used as a template for your own input - copy it
% to a new file with a new name and use it as the basis
% for your article.
%
% NB: the document class 'llncs' has its own and detailed documentation, see
% ftp://ftp.springer.de/data/pubftp/pub/tex/latex/llncs/latex2e/llncsdoc.pdf
%
%%%%%%%%%%%%%%%%%%%%%%%%%%%%%%%%%%%%%%%%%%%%%%%%%%%%%%%%%%%%%%%%%%%


\documentclass[runningheads,a4paper,titlepage]{llncs}

\usepackage{amssymb}
\usepackage[spanish,activeacute]{babel}
\usepackage[utf8]{inputenc}
\setcounter{tocdepth}{3}
\usepackage{graphicx}
\usepackage{url}
\pagestyle{empty}

\urldef{\mailsa}\path|{javierd, asabolansky, avizcaino,
einar}@linti.unlp.edu.ar|    
  
\newcommand{\keywords}[1]{\par\addvspace\baselineskip
\noindent\keywordname\enspace\ignorespaces#1}

\begin{document}
\begin{titlepage}

\title{DJBot: Administrando las salas de PC evitando la consola}
\titlerunning{DJBot: Administrando las salas de PC evitando la consola}


\author{Javier Díaz, Aldo Vizcaino, Alejandro Sabolansky y Einar Lanfranco}
\authorrunning{DJBot: Interfaz web para salas de PC}
\institute{LINTI\\
Laboratorio de Investigaci\'on en Nuevas Tecnolog\'ias Inform\'aticas,\\
calle 50 y 120, La Plata,  Argentina\\
\mailsa\\
\url{http://www.linti.unlp.edu.ar}}

\maketitle


\begin{abstract}
La necesidad de optimizar las tareas de mantenimiento y uso remoto en las salas
de PC de la Facultad de Inform\'atica de la UNLP dio lugar al desarrollo de DJBot:
una aplicaci\'on web realizada en Python que permite ejecutar comandos en
m\'ultiples computadoras en un solo paso. 
En este documento se describen los motivos y el camino recorrido para llegar a
la versión de la aplicación disponible hoy en día.
DJBot se encuentra liberado como Software Libre para que cualquiera que desee
pueda utilizarlo y contribuir en el proyecto.
Desde su desarrollo evolucion\'o de ser una simple herramienta ejecutable en 
la l\'inea de comandos a convertirse en una plataforma de administraci\'on
centralizada que simplifica el mantenimiento, actualiza los equipos 
y permite ejecutar tareas programadas en la interfaz web.  

\bigskip 

%\keywords{Botnet, Django, Fabric, Python, Redis}
\noindent \textbf{Palabras clave: } Botnet, Django, Fabric, Python, Redis
\end{abstract}

\end{titlepage}


\section{El contexto}
\noindent La Facultad de Informática de la Universidad Nacional de La Plata (UNLP)
actualmente cuenta con tres salas de PC, las cuales suman aproximadamente 80 equipos, 
que habitualmente se utilizan para dictar clases de las diferentes
c\'atedras, realizar competencias, jornadas y otras actividades extracurriculares. 

Todos los equipos disponibles cuentan con dos sistemas operativos instalados:
Microsoft Windows 7 y la distribución de GNU/Linux que se desarrolla en la Facultad,
Lihuen GNU/Linux\cite{lihuen}, y queda a criterio de las cátedras y de los
alumnos cuál utilizar.

Las tres salas se encuentran conectadas a la red troncal de la Facultad con
acceso restringido desde el exterior, pero utilizando direccionamiento IPv4
público, lo que permite que los equipos sean identificados desde Internet.

El grupo de trabajo que conforman los autores de este documento, además de
encontrarse a cargo de la administración y mantenimiento de las tres salas de PC, 
tiene acceso a los routers y demás dispositivos de interconexión dado que
también es el grupo responsable del mantenimiento del enlace troncal de datos de
la Facultad.


\section{La problemática}

\noindent Se pueden identificar dos cuestiones principales para resolver.

Por un lado, la problemática relacionada con las cuestiones rutinarias. La gran
cantidad de actividades académicas que requieren el uso diario de las salas hace
que la disponibilidad de espacio temporal para realizar tareas de mantenimiento en
cada computadora sea casi nulo. Peor a\'un es el escenario que se presenta
habitualmente donde los trabajos no se hacen sobre un equipo sino que se quieren
realizar tareas en todos los equipos de la sala en un solo paso. 
Por ejemplo, un cambio de configuración o la instalación de
un software requerido por alg\'un usuario debe realizarse sobre todos los
equipos de la sala.

Junto con la  problemática planteada en relación a las tareas de mantenimiento,
surgió la inquietud de cómo se podrían utilizar los equipos de la sala en los
tiempos en que el equipamiento está ocioso para realizar alguna tarea específica
como por ejemplo una prueba distribuida de carga a un servidor web.  

Hasta el día de hoy, para realizar cualquier tipo de tareas era necesario
ejecutar el software en forma manual por el operador en cada una de las máquinas
involucradas mediante la interacción directa con cada uno de los equipos.

Para posibilitar y facilitar la realización en tiempo y forma de estas tareas se ha
desarrollado DJBot, el proyecto que aqu\'i se presenta.


\section{La propuesta}

\noindent Como respuesta a las problemáticas planteadas en la sección anterior se ha
desarrollado una aplicación de administración simplificada y centralizada de
terminales GNU/Linux la cual se controla mediante una interfaz web. 

Todo ello fue desarrollado siguiendo el principio DRY (Dont Repeat Yourself)
mediante la reutilización de componentes de software libre. 
A modo de resumen, se puede decir que DJBot fue realizado
utilizando Python como lenguaje de programación, junto con diversos componentes y
bibliotecas, como el framework Django para la interfaz de usuario, la
biblioteca Fabric y el protocolo SSH, entre otros.

\section{El camino}

\noindent En un primer intento de solución, se utilizó Parallel SSH
(PSSH)\cite{parallelssh}, una herramienta que permite entre otras cosas,
realizar acciones mediante la
ejecuci\'on de comandos y copiar archivos en diferentes computadoras en
paralelo. PSSH se encuentra en los repositorios
oficiales de la distribuci\'on Debian de GNU/Linux, lo que transitivamente hace
que tambi\'en est\'e disponible en Lihuen GNU/Linux y simplifica la puesta en
producción del entorno.

Por su forma de funcionamiento, esta aplicación requiere que las direcciones IP
de las máquinas a las cuales les solicita acciones se encuentren listadas en un archivo de texto.
Con este archivo se ejecutan conexiones por SSH a 
cada una de las direcciones listadas.
Por los mecanismos de protección propios de SSH\cite{manssh} aparecieron restricciones:
\begin{itemize}
 \item  SSH funciona utilizando clave p\'ublica-privada. Los clientes --es decir,
todas las máquinas cuya IP se encuentra en el archivo-- deben contar con la
clave p\'ublica del que se conectar\'a para autorizarlo en el archivo ~/authorized\_keys. 
Esto se solucionó propagando la clave pública a todos los equipos de todas las salas.
 \item  SSH mantiene un archivo de confianza (~/.ssh/known\_hosts) donde guarda
IP y clave pública de todos los pares con los con que dialoga en algún momento.
Si ante un nuevo intento conexión el par no coincide, el servidor rechaza la
conexión. Hist\'oricamente las m\'aquinas de la Facultad estuvieron configuradas
mediante un servicio de asignación dinámica de direcciones, de manera que la
asignaci\'on de las direcciones IP se completaba en forma aleatoria. Con
el servicio de SSH en funcionamiento, en cambio, el servicio de DHCP se
modific\'o para que cada computadora pasara a tener una \'unica IP fija y
definitiva. As\'i, cuando se accedi\'o por primera vez a las computadoras
mediante SSH, se pudo generar el registro de identificaci\'on de cada m\'aquina
que asocia la computadora con su IP. 
\end{itemize}

Una vez concluida esta etapa, se contaba con un software funcional que
cubría en forma parcial las expectativas planteadas ya que permitía la 
administración remota de las salas. Sin embargo, la herramienta seguía sin 
cumplir las expectativas en su totalidad, tanto las originales como las que
fueron surgiendo a medida que el desarrollo avanzaba.

En una segunda etapa del proyecto, surgió el desarrollo de una interfaz web 
como respuesta a la necesidad de que las tareas de mantenimiento y  
soporte pudieran ser realizadas desde cualquier lugar y por personal que no necesariamente tenga acceso a
la consola de administración. El hecho de que la implementaci\'on de PSSH exija la
ejecuci\'on del int\'erprete de comandos BASH del sistema operativo GNU/Linux
sumaba direccionamientos indirectos al proceso retrasando las tareas.  Por esto, se
comenzaron a estudiar alternativas y apareció la biblioteca
Paramiko\cite{paramiko} --utilizada por PSSH-- que no presenta la desventaja de
los direccionamientos indirectos, pero solo permite establecer las conexiones de
SSH de forma simple, lo cual dificulta el envio de órdenes a los clientes.

Por ello se descartó, y buscando una alternativa se optó por Fabric\cite{fabric},
una biblioteca que re\'une las características más útiles de PSSH y que utiliza Paramiko
para realizar las conexiones SSH.

En resumen, mediante Fabric se facilita la configuración de las tareas que se realizan en los equipos de las salas, facilitando la forma de indicar  en qu\'e terminales queremos ejecutar tareas, permitiendo utilizar
distintos archivos de autenticaci\'on SSH y brindando la opci\'on de elegir entre
distintos usuarios, entre otras opciones.

Como última instancia restaba decidir cómo desarrollar la parte web; hoy es muy
difícil pensar en desarrollar una aplicación web utilizando el lenguaje
únicamente sin utilizar algún framework de desarrollo que acorte y simplifique
el proceso mediante la provision de componentes
genéricos como ser filtros de seguridad, plugins de autenticación, acceso a
la bases de datos o mecanismos de templates para el desarrollo de las interfaces
de usuario. Si bien existen numerosos frameworks disponibles para el desarrollo,
el elegido en este caso fue Django\cite{django}, uno de los frameworks más
difundidos y utilizados en el mundo del software libre; Seleccionado en este caso, 
fundamentalmente porque tiene la particularidad al igual que la biblioteca Fabric, de estar codificado en el
lenguaje de programaci\'on Python\cite{python}.

Una caracter\'istica particular de DJBot es su modo de funcionamiento. Las redes
de zombis, m\'as conocidas como ``botnet'' en el 'area de la seguridad
inform\'atica, son redes formadas por equipos que realizan tareas automatizadas,donde hay un controlador principal, que generalmente a través de Internet que indica qué hacer a
una gran cantidad de equipos ``zombis'' que siguen las ordenes sin preguntar el
por qué. El dise\~no particularmente \'util y simple que permite el framework de
desarrollo Django, sumado al concepto b\'asico de las ``botnets'', dieron lugar al
nombre de esta aplicación: DJBot\cite{djbot}. Como las partes que conforman su
nombre lo indican, ``DJ'' hace mención a Django y ``Bot'' hace referencia al
funcionamiento similar al de una ``botnet''. As\'i, DJBot es un sistema que
integra manejo de conexiones SSH con una interfaz web.




\section{El modelo de datos}

\noindent El modelo de datos representado en la Figura 1 est\'a compuesto por cuatro
entidades: Aula, Computadora, Tarea y Configuraciones.

\begin{figure}
\centering
\includegraphics[height=6.2cm]{modelo_de_datos}
\caption{Clases del modelo de datos}
\label{fig:Modelo de datos}
\end{figure}


La entidad Aula est\'a conformada por m\'ultiples computadoras y presenta las
siguientes caracter\'isticas principales:
\begin{enumerate}
\item{nombre del aula,}
\item{usuario,}
\item{direcci\'on IP de una m\'aquina intermediaria,}
\item{nombre de la placa de red de la m\'aquina intermediaria, y}
\item{direcci\'on de red.}
\end{enumerate}
El nombre sirve para identificar el aula. La identificaci\'on mediante usuario
permite realizar el inicio de sesi\'on en cada computadora del aula. La
direcci\'on IP y el nombre de la interfaz de la m\'aquina intermediaria, que por
lo general es un router, permiten encender las computadoras de cada aula a
trav\'es de la red (wake on LAN). La direcci\'on de red, por su parte, se
utiliza para asignar todas las m\'aquinas encendidas dentro del rango de red
indicado al aula espec\'ifica.

La entidad Computadora incluye los valores:
\begin{enumerate}
\item{nombre de la computadora,}
\item{direcci\'on MAC,}
\item{direcci\'on IP, y}
\item{aula a la que pertenece.}
\end{enumerate}

La entidad Tarea se define a trav\'es de:
\begin{enumerate}
\item{nombre de la tarea,}
\item{conjunto de instrucciones, y}
\item{archivo opcional.}
\end{enumerate}

La lista de instrucciones completa la tarea en su totalidad. El archivo opcional
permite compartir informaci\'on entre todas las computadoras del aula, y se
utiliza cuando resulta m\'as sencillo que hacer uso de las instrucciones de
comandos.

En la actualidad, la entidad Configuraciones se utiliza para definir valores
predeterminados para las dem\'as entidades. La clave SSH de las aulas, por
ejemplo, est\'a definida bajo la entidad Configuraciones. Sin embargo, en el
futuro se planea configurar claves SSH espec\'ificas para cada aula en
particular. As\'i, la entidad Configuraciones quedar\'ia disponible para cubrir
cualquier necesidad de definici\'on de valores predeterminados.


\section{La optimización de DJBot}
\noindent El desarrollo de DJBot, es decir, la integraci\'on del framework con las
bibliotecas y con el modelo de datos que se describi\'o anteriormente,  se
complet\'o con el
agregado de la interfaz web para lograr practicidad y flexibilidad en este
trabajo.

La ejecuci\'on de tareas como la actualizaci\'on del
sistema y la instalaci\'on de aplicaciones nuevas, por ejemplo, implican tiempos
de trabajo que al ser procesados desde la interfaz web de DJBot devolver\'ian un
error por tiempo de espera agotado. La utilizaci\'on del buffer
Redis\cite{redis} para almacenar las tareas solicitadas y los resultados
devueltos evita, entonces, que el navegador quede como si estuviera cargando
algo generándose
retrasos innecesarios. Con la ayuda de la biblioteca RQWorker\cite{rqworker},
Python se comunica con Redis y de esta manera las tareas se colocan en una cola
de
espera y son ejecutadas a su debido tiempo seg\'un el orden de solicitud. Esto
independiza la interfaz web de DJBot para que el administrador pueda seguir interactuando con la misma sin tener demoras.


\section{Los casos de uso}


\noindent DJBot ya ha sido utilizado en la administración de las salas de PC de las
facultad en diversas ocasiones, facilitando y automatizando las distintas
tareas, entre las que podemos mencionar:

\begin{itemize}
 \item Instalación de software solicitado por las cátedras para la realización
de las actividades académicas.
 \item Actualización del sistema operativo ante actualizaciones de funcionalidad
y seguridad de las distintas herramientas instaladas.
 \item Despliegue de una arquitectura para la realización de un test de stress
sobre una aplicación web. 
\end{itemize}

\section{La liberación}
\noindent DJBot está liberado como software libre bajo licencia GPL en su versión 3, una
licencia que garantiza los principios del software libre permitiendo el libre
uso, distribución y modificación del software a gusto de cualquiera que así lo
desee.

Actualmente el código del proyecto puede encontrarse en los servidores de
GitHub, uno de los repositorios de software libre más utilizados en la 
actualidad. Por como funciona GitHub, el lector puede descargarlo libremente
desde allí; si es para usarlo
puede hacerlo en forma anónima, pero en cambio si quiere generar algún aporte va
a
necesitar generarse un usuario en el sistema que le permita luego integrar su
contribución al desarrollo.

Para acceder al desarrollo, solamente es necesario apuntar un navegador web a
https://github.com/krahser/djbot.


\section{El trabajo a futuro}
\noindent Despu\'es de varios meses de trabajo y de tener la aplicaci\'on web funcionando,
son varios los beneficios obtenidos y tambi\'en son muchas las mejoras que están
planificadas para  ser aplicadas en el corto plazo. 


Las cuestiones m\'as importantes en las que se continuará trabajando involucran:

\begin{itemize}
\item{Mejora de la interfaz gr\'afica para que resulte m\'as simple de
utilizar.}
\item{Aplicación de  mecanismos asincrónicos de notificación a través del uso de
AJAX
para agregar interacción a DJBot (ya se han realizado algunas pruebas con la
biblioteca Dajaxice).}
\item{Mejora de la seguridad del sistema mediante el uso de claves SSH
independientes por sala.}
\item{Agregado de  un emulador de consola a la interfaz gr\'afica.}
\item{Mejora del funcionamiento del buffer.}
\item{Comprobaci\'on de la funci\'on de encendido de computadoras por red. Este
ítem es necesario para evitar el tener que desplazarse hasta las salas para
iniciar los equipos.}
\item{Implementaci\'on de la funci\'on de descubrimiento de computadoras
encendidas disponibles para trabajar en tiempo real.}
\item{Carga autom\'atica de los valores de las computadoras que conforman una
sala mediante una funci\'on de descubrimiento por red.}
\end{itemize}

\section{Las conclusiones}
\noindent La disponibilidad limitada de las salas de PC de la Facultad de Inform\'atica
motivó que desde el LINTI se desarrollara una aplicaci\'on web para poder
realizar el trabajo de mantenimiento y administración en tiempo y forma sin
interferir con la utilización académica habitual de las distintas salas de PC.
Además de solucionar
la problemática original, DJBot aportó soluciones colaterales, ya que no sólo
permite ejecutar
tareas y copiar archivos en m\'as de una m\'aquina al mismo tiempo, sino que
ahora este trabajo no requiere de responsables expertos en la materia para poder
hacerlo, 
ya que cualquiera puede invocar una tarea preprogramada desde la interfaz web.
El acceso también es más simple; debido a que es una aplicación web, es posible
acceder a la
misma mediante un navegador web y así, sin más, tener acceso a todas las terminales salas de
PC.

Practicidad y flexibilidad son apenas dos de los beneficios que brinda esta
herramienta. Por otro lado, gracias a que ahora las computadoras de la Facultad
est\'an disponibles para ser utilizadas para realizar otras tareas en cualquier
momento 
momento en que se encuentren ociosas y desde sitios remotos. Tambi\'en se
comenzaron a realizar tareas de
investigaci\'on, como pruebas de estabilidad y
carga distribuida de servicios web.

La práctica y experiencia de uso que se ha tenido hasta el momento son aspectos
motivadores para continuar mejorando DJBot y seguir enfrentando los desafíos
diarios con creatividad y precisi\'on. DJBot es tan solo un ejemplo del amplio
abanico de
soluciones que est\'an a la espera de ser creadas.

\begin{thebibliography}{4}
%\bibitem{} , \url{}
\bibitem{parallelssh} Parallel SSH, \url{http://code.google.com/p/parallel-ssh/}
\bibitem{lihuen} GNU/Linux Lihuen , \url{http://www.lihuen.linti.unlp.edu.ar}
\bibitem{dhcp} Dinamic Host Configuration Protocol,
\url{http://linux.die.net/man/5/dhcpd.conf}
\bibitem{manssh} SSH Known Hosts,
\url{http://www.linuxmanpages.com/man1/ssh.1.php}
\bibitem{paramiko} Paramiko, \url{https://github.com/paramiko/paramiko}
\bibitem{fabric} Fabric, \url{http://docs.fabfile.org/en/1.6/}
\bibitem{django} Django, versi\'on 1.4.5, \url{https://www.djangoproject.com/}
\bibitem{python} Python, versi\'on 2.7, \url{http://www.python.org/}
\bibitem{redis} Redis, \url{http://redis.io/}
\bibitem{rqworker} Django rqworker, \url{https://github.com/ui/django-rq/}
\bibitem{djbot} DJBot, \url{https://github.com/krahser/djbot}
\end{thebibliography}
\end{document}
